% Use this template to write your solutions to COS 423 problem sets

\documentclass[12pt]{article}
\usepackage[utf8]{inputenc}
\usepackage{amsmath, amsfonts, amsthm, amssymb, algorithm, graphicx, mathtools,xfrac}
\usepackage[noend]{algpseudocode}
\usepackage{fancyhdr, lastpage}
\usepackage[vmargin=1.20in,hmargin=1.25in,centering,letterpaper]{geometry}
\setlength{\headsep}{.50in}
\setlength{\headheight}{15pt}

% Landau notation
\DeclareMathOperator{\BigOm}{\mathcal{O}}
\newcommand{\BigOh}[1]{\BigOm\left({#1}\right)}
\DeclareMathOperator{\BigTm}{\Theta}
\newcommand{\BigTheta}[1]{\BigTm\left({#1}\right)}
\DeclareMathOperator{\BigWm}{\Omega}
\newcommand{\BigOmega}[1]{\BigWm\left({#1}\right)}
\DeclareMathOperator{\LittleOm}{\mathrm{o}}
\newcommand{\LittleOh}[1]{\LittleOm\left({#1}\right)}
\DeclareMathOperator{\LittleWm}{\omega}
\newcommand{\LittleOmega}[1]{\LittleWm\left({#1}\right)}

% argmin and argmax
\newcommand{\argmin}{\operatornamewithlimits{argmin}}
\newcommand{\argmax}{\operatornamewithlimits{argmax}}

\newcommand{\calP}{\mathcal{P}}
\newcommand{\Z}{\mathbb{Z}}
\newcommand{\R}{\mathbb{R}}
\newcommand{\Exp}{\mathbb{E}}
\newcommand{\Q}{\mathbb{Q}}
\newcommand{\sign}{\mathrm{sign\ }}
\newcommand{\abs}{\mathrm{abs\ }}
\newcommand{\eps}{\varepsilon}
\newcommand{\zo}{\{0, 1\}}
\newcommand{\SAT}{\mathit{SAT}}
\renewcommand{\P}{\mathbf{P}}
\newcommand{\NP}{\mathbf{NP}}
\newcommand{\coNP}{\co{NP}}
\newcommand{\co}[1]{\mathbf{co#1}}
\renewcommand{\Pr}{\mathop{\mathrm{Pr}}}

% theorems, lemmas, invariants, etc.
\newtheorem{theorem}{Theorem}
\newtheorem{lemma}[theorem]{Lemma}
\newtheorem{invariant}[theorem]{Invariant}
\newtheorem{corollary}[theorem]{Corollary}
\newtheorem{definition}[theorem]{Definition}
\newtheorem{property}[theorem]{Property}
\newtheorem{proposition}[theorem]{Proposition}

% piecewise functions
\newenvironment{piecewise}{\left\{\begin{array}{l@{,\ }l}}
{\end{array}\right.}

% paired delimiters
\DeclarePairedDelimiter{\ceil}{\lceil}{\rceil}
\DeclarePairedDelimiter{\floor}{\lfloor}{\rfloor}
\DeclarePairedDelimiter{\len}{|}{|}
\DeclarePairedDelimiter{\set}{\{}{\}}

\makeatletter
\@addtoreset{equation}{section}
\makeatother
\renewcommand{\theequation}{\arabic{section}.\arabic{equation}}

% algorithms
\algnewcommand\algorithmicinput{\textbf{INPUT:}}
\algnewcommand\INPUT{\item[\algorithmicinput]}
\algnewcommand\algorithmicoutput{\textbf{OUTPUT:}}
\algnewcommand\OUTPUT{\item[\algorithmicoutput]}


% Formating Macros

\pagestyle{fancy}
\lhead{\sc \hmwkClass $\;\;\cdot \;\;$ \hmwkSemester $\;\;\cdot\;\;$
Problem \hmwkAssignmentNum.\hmwkProblemNum}
%\chead{\sc Problem \hmwkAssignmentNum.\hmwkProblemNum}
%\chead{}
\rhead{\em \hmwkAuthorName\ $($\hmwkAuthorID$)$}
\cfoot{}
\lfoot{}
\rfoot{\sc Page\ \thepage\ of\ \protect\pageref{LastPage}}
\renewcommand\headrulewidth{0.4pt}
\renewcommand\footrulewidth{0.4pt}

\fancypagestyle{fancycollab}
{
    \lfoot{\em Collaborators: \hmwkCollaborators}
}

\fancypagestyle{problemstatement}
{
    \rhead{}
    \lfoot{}
}

%%%%%% Begin document with header and title %%%%%%%%%%%%%%%%%%%%%%%%%

\begin{document}

%%%%%% Header Information %%%%%%%%%%%%%%%%%%%%%%%%%%%%%%%%%%%%%%%%%%%

%%% Shouldn't need to change these
\newcommand{\hmwkClass}{COS 423}
\newcommand{\hmwkSemester}{Spring 2013}

%%% Your name, in standard First Last format
\newcommand{\hmwkAuthorName}{Parker Watts}
%%% Your NetID
\newcommand{\hmwkAuthorID}{watts}

%%% The problem set number (just the number)
\newcommand{\hmwkAssignmentNum}{0}

%%% The problem number (just the number)
\newcommand{\hmwkProblemNum}{1}

%%% A list of your collaborators' NetIDs, separated by ", ".
%%% You can use a new line ("\\") in the middle to prevent a long
%%% list from overflowing.
\newcommand{\hmwkCollaborators}{Rafael Zuniga}
%%% Sets the collaborator list to appear on the first page
\thispagestyle{fancycollab}

%%%%%%% begin Solution %%%%%%%%%%%%%%%%%%%%%%%%%%%%%%%%%%%%%%%%%%%%%

%%%%%%% begin Problem %%%%%%%%%%%%%%%%%%%%%%%%%%%%%%%%%%%%%%%%%%%%%%

\section{Burrows-Wheeler Data Compression Algorithm}
 Implement the Burrows-Wheeler data compression algorithm. This revolutionary algorithm outcompresses gzip and PKZIP, is relatively easy to implement, and is not protected by any patents. It forms the basis of the Unix compression utility bzip2.

The Burrows-Wheeler compression algorithm consists of three algorithmic components, which are applied in succession:

    Burrows-Wheeler transform. Given a typical English text file, transform it into a text file in which sequences of the same character occur near each other many times.

    Move-to-front encoding. Given a text file in which sequences of the same character occur near each other many times, convert it into a text file in which certain characters appear more frequently than others.

    Huffman compression. Given a text file in which certain characters appear more frequently than others, compress it by encoding frequently occurring characters with short codewords and rare ones with long codewords. 

Step 3 is the one that compresses the message: it is particularly effective because Steps 1 and 2 result in a text file in which certain characters appear much more frequently than others. To expand a message, apply the inverse operations in reverse order: first apply the Huffman expansion, then the move-to-front decoding, and finally the inverse Burrows-Wheeler transform. Your task is to implement Burrows-Wheeler and move-to-front components efficiently. 

%%%%%%% end Problem %%%%%%%%%%%%%%%%%%%%%%%%%%%%%%%%%%%%%%%%%%%%%%%%

\newpage

%%%%%%% begin Problem %%%%%%%%%%%%%%%%%%%%%%%%%%%%%%%%%%%%%%%%%%%%%%

\section{Circular Suffix Array}
 To efficiently implement the key component in the Burrows-Wheeler transform, you will use a fundamental data structure known as the circular suffix array, which describes the abstraction of a sorted array of the $N$ circular suffixes of a string of length $N$. We define index[$i$] to be the index of the original suffix that appears $i$th in the sorted array.
 
 

%%%%%%% end Problem %%%%%%%%%%%%%%%%%%%%%%%%%%%%%%%%%%%%%%%%%%%%%%%%

\section{Mathematical Formulation}
We must create a $N$ by $N$ array of characters, beginning with element $0$ of the array as the original string we are analysing, then element $1$ being the original string shifted one element to the left, etc.

We then sort the rows of the array, and then return the $N$th column 
of this array, preceded by the index in which the original string 
appears in the sorted array.

\section{An Efficient Solution}

Our algorithm casts the original string as an Array List of Characters. 
Since a string is effectively an array of characters, we create the 
$N$ by $N$ array of characters as an array of strings. We fill this 
array of strings by removing the first element of the Array List, 
adding it to the End of the Array List, then casting the current 
elements of the Array List to a string and adding it to the array of 
strings, then repeating the process until the array is full. We also 
append the original index number to each string, to build the index 
array later.

We sort the string array, and then use the int values we appended to
those strings to build the necessary index array. Once this is
done, it is trivial to return the $N$the column.


\begin{algorithm}[H]
\caption{Circular Suffix Array.}
\begin{algorithmic}
    \Procedure{CircularSuffixArray}{String $S$}
        \State $ArrayList \gets$ \Call{MakeArrayList}{$S$}
        \State $Suffix \gets$ \Call{MakeArray}{$String[]$}
        \For{$i < Length of S$}
        		\State \Call{Shift Left}{$ArrayList$}
            \State \Call{add}{$Suffix, String \gets ArrayList \gets String${$i$}}
        \EndFor
        \State $Index \gets$ \Call{MakeArray}{$int[]$}
        
        \For{$String \in Suffix$}
            \State $Index \gets$ \Call{add}{$Integer \in String$}
        \EndFor
    \EndProcedure
\end{algorithmic}
\end{algorithm}


\section{Correctness}

\begin{proposition}
The algorithm creates an $N$ by $N$ array of characters, iterated correctly, sorted, and with a corresponding index of length $N$
\end{proposition}

\begin{proof}
We begin with an input consisting of a single uncompressed string, and from this we must output the corresponding string which is ready for compression, as well as the integer from the index array pertaining to 0.
Due to the inherent type of ArrayList, moving an item on the list from the head to the tail shifts the entire list one space to the left. By both keeping track of the length of the string, and appending an int value to the end of the strings, we can sort the string array using the Arrays.sort quicksort function, and then recover that int value to construct the index. Keeping the length variable is also what allows us to pass the $N$th column of the array as a string to the Burrows Wheeler Class. The index contains one element per row of the table, representing the original position of that string in order before the sorting occurred, and thus has $N$ elements.
\end{proof}

\section{Analysis}

\begin{proposition}
\label{numq}
The Circular Suffix Array function operates in $3NlogN$ time, and utilizes $N$ squared plus $R$ space.
\end{proposition}

\begin{proof}
Because we use the Array List to construct the circular suffix array, it takes only $N$ squared plus  R time. Since we are creating an $N$ by $N$ array by definition of the class, we cannot possibly go lower than $N$ squared. We avoid using the substring method for string, since strings are immutable and creating the next string for the suffix array would involve taking the previous string, creating two more strings from it, and then merging those strings into a new string, giving a total space usage of $3N$ squared, which would be unacceptable. Since the Array List is mutable, it costs only R additional space from the Array List when constructing the suffix array. Constructing the index also uses $N$ time, since appending the index on the end of the string while maintaining the length value means that extracting those strings takes N time, with a marginal increase based on the number of digits in the int. the sorting of the arrays themselves takes $NlogN$ as is intrinsic for quicksort.
\end{proof}



%%%%%%% end Solution %%%%%%%%%%%%%%%%%%%%%%%%%%%%%%%%%%%%%%%%%%%%%%%

%%%%%%% begin Problem %%%%%%%%%%%%%%%%%%%%%%%%%%%%%%%%%%%%%%%%%%%%%%

\section{Burrows-Wheeler Transform}
 To efficiently implement the key component in the Burrows-Wheeler transform, you will use a fundamental data structure known as the circular suffix array, which describes the abstraction of a sorted array of the $N$ circular suffixes of a string of length $N$. We define index[$i$] to be the index of the original suffix that appears $i$th in the sorted array.
 
 

%%%%%%% end Problem %%%%%%%%%%%%%%%%%%%%%%%%%%%%%%%%%%%%%%%%%%%%%%%%

\section{Mathematical Formulation}
For the transform function, we simply create a circular suffix array (as above), return the $N$th column, then return an index value such that $index[i]$ == $0$

The inverse transform is slightly trickier. With an input of an index of the original string and the transformed string, we recreate the original. First, we sort the transformed array to regain the first column of the circular suffix array, then we recreate the index array by determining the relative positions of the iterations of characters. With this reconstructed index, we follow along it iteratively to determine what the correct sequence of characters was.

\section{An Efficient Solution}

Our transform function consists almost entirely of our earlier circular suffix array's formation, and has such has little to expand on here.

The inverse function, after retaining the correct values for the transformed string, the sorted string, and the original string's index, we construct a hashmap with each character that appears in the string as a key, and the corresponding value being an Array List of integers which represent the positions that each of those characters occupy in the transformed string. We use this hashmap to build the next array, starting at element $0$, and using the smallest possible index value for each given character in the sorted string.

With that done, we find the corresponding index for the original string in order to get the first and last character, and then follow the pattern of next characters outlined in the index array to reconstruct the string.

\begin{algorithm}[H]
\caption{Burrows Wheeler Transform}
\begin{algorithmic}
    \Procedure{Transform}{}
        \State $String S \gets$ \Call{Binary Input}{$String$}
        \State $Suffix \gets$ \Call{Circular Suffix Array}{$S$}
        \State $String Transformed \gets$ \Call{Circular Suffix Array}{$Nth Column$}
        \State $int Index \gets$ \Call{Get Original Suffix}{$Suffix$}
        \State $Write \gets$ \Call{$Index$ + $Transformed$}{}
    \EndProcedure
\end{algorithmic}
\end{algorithm}

\begin{algorithm}[H]
\caption{Burrows Wheeler Inverse Transform}
\begin{algorithmic}
    \Procedure{Inverse Transform}{}
        \State $int- BaseIndex \gets$ \Call{Binary Input}{$int$}
        \State $String- TransformedString \gets$ \Call{Binary Input}{$String$}
        \State $String- SortedString \gets$ \Call{Sort}{$TransformedString$}
		\State HashMap$\gets$ \Call{MakeHashMap}{$char, Array List$}      
        \For{$chars \in TransformedString$}
        		\State \Call{Add to $char$'s Array List}{Index of $char$}
        \EndFor
		\While{$next[] \gets !Full$}
			\State $next[] \gets$ \Call{Hashmap $/gets ArrayList$}{Element at $currentChar$}
		\EndWhile        
        \State $String- OrininalString${}
        \For{$Sequence of ints in next[] starting at 0$}
            \State $OriginalString \gets$ \Call{add}{$SortedString${charAt{$currentInt$}}}
        \EndFor
        \State $Write \gets$ \Call{$OriginalString$}{}
    \EndProcedure
\end{algorithmic}
\end{algorithm}


\section{Correctness}

\begin{proposition}
These algorithms take a string and transform in to and back from different sequences based on a baseline integer.
\end{proposition}

\begin{proof}
The proof for the Transform function is effectively the same as that for the Circular Suffix Array with a binary string as input and the output being both the int and string produced by Circular Suffix Array. 

For the Inverse Transform function, we begin with a transformed string and the index in which the original string appeared in a circular suffix array. We first use the fact that the next[] value for two strings, i and j, if next[i+1] == j, then the sorted character of i equals the transformed character of j, to build the index in successive order. Once we have that order and can start at 0, we simply iterate through the transformations (next[next[next...[first]]]) to gain the correct sequence of characters.
\end{proof}

\section{Analysis}

\begin{proposition}
\label{numq}
The Transform function operates in $5NlogN$ time, and the Inverse Transform function operates in $4NlogN$ time.
\end{proposition}

\begin{proof}
Transform function is literally the Circular Suffix Array with an additional 2$N$ operations for fetching the final column and collecting the correct index. Fetching the column involves creating a string out of $N$ chars, one from each column, to form a single string, so it is self-evidently in $N$ time. The method for obtaining the index related to 0 can often be sublinear (in the "ABRACADABRA!" example, it completes after only one call), but in the worse case (if the desired element is at the end of the index array), it operates in $N$ time. With the proof above in mind, the total running time of the transform function, plus the circular suffix array, is $5NlogN$.

The Inverse Transform's HashMap can be created in $N$ time since adding an element to the has takes constant time, as a necessary property of hash classes, and we are adding $N$ elements. Its existance makes constructing the next array consume $NlogN$ time, since each character accesses the Hashmap in $logN$ time, and we must access the map a total of $N$ times, one for each int in the index array. After the index array has been fully filled in, creating the actual original string using the next array takes an additional $2N$, in a process similar to locating the correct index in the transform function (which also may be sublinear). One the index pertaining to 0 is located, following the index in a next[next[next...[first]]] fashion, which takes $N$ time, since it will pass through each element in the index exactly once.
\end{proof}



\section{Move-To-Front}
 The main idea of move-to-front encoding is to maintain an ordered sequence of the characters in the alphabet, and repeatedly read in a character from the input message, print out the position in which that character appears, and move that character to the front of the sequence.
 
 

%%%%%%% end Problem %%%%%%%%%%%%%%%%%%%%%%%%%%%%%%%%%%%%%%%%%%%%%%%%

\section{Mathematical Formulation}
We maintain an array with all the ascii symbols in it in an ordered fashion depending on the input given. We read N characters from the input, where N is the length of the given String, and we keep track of the current position of that character in out ascii array.

\section{An Efficient Solution}
For encoding or decoding, we keep an array of size 256 that holds all the ascii characters in their original order. For encoding, we then proceed to read the input character by character. For every character, we figure out the current index of that character in the ascii array by starting a linear search from ascii[255] down. Once we find the character in the ascii array, we print out the index, and then we shift the rest of the elements one index forward. Then we finally just make the 0th element be the current character.

For decoding, for every character we read, we get the int value of that character and store it in a variable, lets call it temp. We then use this value and we shift all elements before ascii[temp] one index forward. We then proceed to assign ascii[0] the original character at ascii[temp] and we print it out using BinaryStdOut.

\section{Correctness}

\begin{proposition}
These algorithms takes a string and maintains an ordered sequence of characters in the alphabet where characters in the input appear at the front of the sequence.
\end{proposition}

\begin{proof}
If we maintain an array with the ascii characters and every time we read in a character we move that character into the 0th element of the ascii array, the algorithm will create an ordered sequence of the alphabet were characters that are used more often are at the front of the sequence
\end{proof}

\section{Analysis}

\begin{proposition}
\label{numq}
Both encoding and decoding take R*N in the worst case where R is the size of the alphabet and N is the length of the String. But in practice, we can have a smaller alphabet and for every character we might only need to go through 26 characters, a low number which could be considered constant. Because of this, we can have a run time proportional to N + R.
The space complexity is proportional to N + R

\end{proposition}

\begin{proof}
If we were to get an input where none of the characters repeat, the algorithm would go through the entire ascii array for every character it reads in, resulting in a running time of R*N. In a typical English text, there will be a lot of repetition and certain characters will appear much more than others. Those characters will be found even more quickly the next time they show up. So in practice, we basically have a smaller alphabet to work with and for every character we might only need to go through 26 characters, a low number which could be considered constant.


To store the information, we only use an array of size R and we store each character only once, resulting in a space complexity proportional to N + R
\end{proof}


%%%%%%% end Solution %%%%%%%%%%%%%%%%%%%%%%%%%%%%%%%%%%%%%%%%%%%%%%%

\end{document}
